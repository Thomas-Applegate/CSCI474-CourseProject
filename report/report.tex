\documentclass[11pt,letterpaper,conference]{IEEEtran}
\usepackage{amsmath,amssymb,amsfonts}
\usepackage{algorithmic}
\usepackage{graphicx}
\usepackage{textcomp}
\usepackage{xcolor}
\usepackage{float}
\usepackage[english]{babel}
%proper format of quotations
\usepackage{csquotes}
%load biblatex for bibliography processing and reference management
%use the ieee style for citations.
\usepackage[style=ieee]{biblatex}
%prevent bad hboxes when formating urls
\usepackage{xurl}

\MakeOuterQuote{"}

%set graphics path where figures will be loaded from.
\graphicspath{ {images/} }

%load bibliography
%this is the file wehre all citations are stored. Latex will figure out the order
%dynamically when it compiles the document
\addbibresource{refs.bib}

\begin{document}

\title{CSCI474 - Course Project Proposal}

\author{\IEEEauthorblockN{Thomas Applegate}
\IEEEauthorblockA{\textit{Colorado School of Mines} \\
Golden, CO, USA \\
tapplegate@mines.edu}
\and
\IEEEauthorblockN{Kaelyn Boutin}
\IEEEauthorblockA{\textit{Colorado School of Mines} \\
Golden, CO, USA \\
kvboutin@mines.edu}
\and
\IEEEauthorblockN{Gabrielle Hadi}
\IEEEauthorblockA{\textit{Colorado School of Mines} \\
Golden, CO, USA \\
ghadi@mines.edu}
\and
\IEEEauthorblockN{Addison Hart}
\IEEEauthorblockA{\textit{Colorado School of Mines} \\
Golden, CO, USA \\
addisonhart@mines.edu}
\and
\IEEEauthorblockN{Et Griffin}
\IEEEauthorblockA{\textit{Colorado School of Mines} \\
Golden, CO, USA \\
egriffin@mines.edu}
\and
\IEEEauthorblockN{Isabelle Neckel}
\IEEEauthorblockA{\textit{Colorado School of Mines} \\
Golden, CO, USA \\
ineckel@mines.edu}
}

\maketitle

\begin{abstract}
The National Institute of Standards and Technology, NIST, defines fifteen separate tests
for the randomness and unpredictability of random numbers. True Random, Deskewed True
Random and Pseudorandom are the three classes of random numbers that are going to be
compared using the tests outlined by NIST. We plan to use a Python implementation that will
provide us the P-values to compare and contrast to determine the random numbers that are
closest to being truly random. The generation of the random numbers will also be considered
when discussing the choice in relation to cryptography.
\end{abstract}

\begin{IEEEkeywords}
random numbers, pseudorandomness, NIST, python
\end{IEEEkeywords}

\section{Introduction}
Introduction here.

\section{Methods}
To test the randomness of various algorithms, we will use the NIST randomness tests as
implemented in Python \cite{rtestsuite}. The tests require a binary string of indeterminate length as the input.
Each of the fifteen tests will output the P-value, as well as the result of whether the P-value means
that the data can be considered truly random or not. For each of the algorithms that we test, we
will use many samples of equal length to compare their randomness to each other.

\section{Expected Results}
Expected Results here.

\section{Conclusion}
Conclusion here.

\printbibliography[heading=bibintoc, title={References}]

\end{document}
